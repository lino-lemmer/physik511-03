\input{header.tex}

\renewcommand\thesection{Übung \arabic{section}}
\renewcommand\thesubsection{\arabic{section}\alph{subsection}}

\title{physik511: Übungsblatt 03}
\author{Lino Lemmer}

\begin{document}
\maketitle
\section{Weizsäckersche Massenformel}
\subsection{Kinetische energie eines $\alpha$-Teilchens}

Die Differenz der Bindungsenergie des Ausgangskern und denen der entstehenden Kerne geht in die kinetische Energie. Also:

\begin{align*}
    T_\text{kin} =& E_\text{bind}(A, Z) - \del{E_\text{bind}(A-4,Z-2) +
    E_{\text{bind,}\alpha}}
    \intertext{Mit der Bindungsenergie}
    E_\text{bind}(A,Z) =& a_\text{v}A-a_\text{o}A^{\frac23}-a_\text{c}Z\del{Z-1}
    A^{-\frac13}-a_\text{s}\frac{\del{A-2Z}^2}{4A} + a_\text{p}A^{-\frac12}
    \intertext{erhält man}
    =& -4a_\text{v} - \del{A^{\frac23} - \del{A-4}^{\frac23}}a_\text{o} -
    \del{Z\del{Z-}A^{-\frac13} - \del{Z-2} \del{Z-3} \del{A-4}^{-\frac13}} a_\text{c} \\
    & - \del{\frac{\del{A-2Z}^2}{A}-\frac{\del{A-8-2Z}^2}{A-4}}
    \frac{a_\text{s}}4 + \del{A^{-\frac12}-\del{A-4}^{-\frac12}}a_\text{p}
    - E_{\text{bind,}\alpha}
    \intertext{Dabei sind die Konstanten}
    a_\text{v} =& \SI{15.67}{\mega\electronvolt} \\
    a_\text{o} =& \SI{17.23}{\mega\electronvolt} \\
    a_\text{c} =& \SI{0.714}{\mega\electronvolt} \\
    a_\text{s} =& \SI{93.14}{\mega\electronvolt} \\
    a_\text{p} =& \begin{cases}
        \SI{11.2}{\mega\electronvolt} & \text{gg} \\
        0 & \text{ug und gu} \\
        \SI{-11.2}{\mega\electronvolt} & \text{uu}
    \end{cases}
\end{align*}

Ich sehe hier nicht, wo Näherungen gemacht werden können.

\subsection{$\alpha$-Zerfall von $_3^6\text{Li}$}

\begin{align*}
    T_\text{kin}(6,3) &= -4a_\text{v}-\del{6^{\frac23}-2^{\frac23}}a_\text{o}
    -6\cdot6^{-\frac13}a_\text{c}+8a_\text{s}-E_\text{bind,}\alpha \\
    &= \SI{622.24}{\mega\electronvolt}
\end{align*}

Der $\alpha$-Zerfall ist also möglich.

\subsection{$\alpha$-Zerfall von $_{57}^{135}\text{La}$}

\section{Die Masse des Neutrons}
\subsection{Bindungsenergie in Deuterium}
\subsection{Masse des Neutrons}

\end{document}
