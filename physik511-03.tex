% Für Seitenformatierung

\documentclass[DIV=15]{scrartcl}

% Zeilenumbrüche

\parindent 0pt
\parskip 6pt

% Für deutsche Buchstaben und Synthax

\usepackage[ngerman]{babel}

% Für Auflistung mit speziellen Aufzählungszeichen

\usepackage{paralist}

% zB für \del, \dif und andere Mathebefehle

\usepackage{amsmath}
\usepackage{commath}
\usepackage{amssymb}

% für nicht kursive griechische Buchstaben

\usepackage{txfonts}

% Für \SIunit[]{} und \num in deutschem Stil

\usepackage[output-decimal-marker={,}]{siunitx}
\usepackage[utf8]{inputenc}

% Für \sfrac{}{}, also inline-frac

\usepackage{xfrac}

% Für Einbinden von pdf-Grafiken

\usepackage{graphicx}

% Umfließen von Bildern

\usepackage{floatflt}

% Für Links nach außen und innerhalb des Dokumentes

\usepackage{hyperref}

% Für weitere Farben

\usepackage{color}

% Für Streichen von z.B. $\rightarrow$

\usepackage{centernot}

% Für Befehl \cancel{}

\usepackage{cancel}

% Für Layout von Links

\hypersetup{
	citecolor=black,
	colorlinks=true,
	linkcolor=black,
	urlcolor=blue,
}

% Verschiedene Mathematik-Hilfen

\newcommand \e[1]{\cdot10^{#1}}
\newcommand\p{\partial}

\newcommand\half{\frac 12}
\newcommand\shalf{\sfrac12}

\newcommand\skp[2]{\left\langle#1,#2\right\rangle}
\newcommand\mw[1]{\left\langle#1\right\rangle}
\renewcommand \exp[1]{\mathrm e^{#1}}

% Nabla und Kombinationen von Nabla

\renewcommand\div[1]{\skp{\nabla}{#1}}
\newcommand\rot{\nabla\times}
\newcommand\grad[1]{\nabla#1}
\newcommand\laplace{\triangle}
\newcommand\dalambert{\mathop{{}\Box}\nolimits}

%Für komplexe Zahlen

\renewcommand \i{\mathrm i}
\renewcommand{\Im}{\mathop{{}\mathrm{Im}}\nolimits}
\renewcommand{\Re}{\mathop{{}\mathrm{Re}}\nolimits}

%Für Bra-Ket-Notation

\newcommand\bra[1]{\left\langle#1\right|}
\newcommand\ket[1]{\left|#1\right\rangle}
\newcommand\braket[2]{\left\langle#1\left.\vphantom{#1 #2}\right|#2\right\rangle}
\newcommand\braopket[3]{\left\langle#1\left.\vphantom{#1 #2 #3}\right|#2\left.\vphantom{#1 #2 #3}\right|#3\right\rangle}


\renewcommand\thesection{Übung \arabic{section}}
\renewcommand\thesubsection{\arabic{section}\alph{subsection}}

\title{physik511: Übungsblatt 03}
\author{Lino Lemmer}

\begin{document}
\maketitle
\section{Weizsäckersche Massenformel}
\subsection{Kinetische energie eines $\alpha$-Teilchens}

Die Differenz der Bindungsenergie des Ausgangskern und denen der entstehenden Kerne geht in die kinetische Energie. Also:

\begin{align*}
    T_\text{kin} =& E_\text{bind}(A, Z) - \del{E_\text{bind}(A-4,Z-2) +
    E_{\text{bind,}\alpha}}
    \intertext{Mit der Bindungsenergie}
    E_\text{bind}(A,Z) =& a_\text{v}A-a_\text{o}A^{\frac23}-a_\text{c}Z\del{Z-1}
    A^{-\frac13}-a_\text{s}\frac{\del{A-2Z}^2}{4A} + a_\text{p}A^{-\frac12}
    \intertext{erhält man}
    =& -4a_\text{v} - \del{A^{\frac23} - \del{A-4}^{\frac23}}a_\text{o} -
    \del{Z\del{Z-}A^{-\frac13} - \del{Z-2} \del{Z-3} \del{A-4}^{-\frac13}} a_\text{c} \\
    & - \del{\frac{\del{A-2Z}^2}{A}-\frac{\del{A-8-2Z}^2}{A-4}}
    \frac{a_\text{s}}4 + \del{A^{-\frac12}-\del{A-4}^{-\frac12}}a_\text{p}
    - E_{\text{bind,}\alpha}
    \intertext{Dabei sind die Konstanten}
    a_\text{v} =& \SI{15.67}{\mega\electronvolt} \\
    a_\text{o} =& \SI{17.23}{\mega\electronvolt} \\
    a_\text{c} =& \SI{0.714}{\mega\electronvolt} \\
    a_\text{s} =& \SI{93.14}{\mega\electronvolt} \\
    a_\text{p} =& \begin{cases}
        \SI{11.2}{\mega\electronvolt} & \text{gg} \\
        0 & \text{ug und gu} \\
        \SI{-11.2}{\mega\electronvolt} & \text{uu}
    \end{cases}
\end{align*}

\subsection{$\alpha$-Zerfall von $_3^6\text{Li}$}
\subsection{$\alpha$-Zerfall von $_{57}^{135}\text{La}$}

\section{Die Masse des Neutrons}
\subsection{Bindungsenergie in Deuterium}
\subsection{Masse des Neutrons}

\end{document}
